Bienvenu sur cette Architecture Commune de l\textquotesingle{}\mbox{\hyperlink{a00060}{Arcade}}. Projet de 2ème année au sein de Epitech.

Source Tree\+: 
\begin{DoxyCode}{0}
\DoxyCodeLine{include/}
\DoxyCodeLine{├── Api.hpp}
\DoxyCodeLine{└── \mbox{\hyperlink{a00060}{Arcade}}}
\DoxyCodeLine{    ├── ArcadeStruct.hpp}
\DoxyCodeLine{    ├── Core}
\DoxyCodeLine{    │   ├── ICore.hpp}
\DoxyCodeLine{    │   ├── IDisplayModule.hpp}
\DoxyCodeLine{    │   └── IGameModule.hpp}
\DoxyCodeLine{    ├── ECS}
\DoxyCodeLine{    │   ├── IComponent.hpp}
\DoxyCodeLine{    │   ├── IEntity.hpp}
\DoxyCodeLine{    │   ├── IEntityManager.hpp}
\DoxyCodeLine{    │   ├── IEventManager.hpp}
\DoxyCodeLine{    │   ├── ISystem.hpp}
\DoxyCodeLine{    │   └── ISystemManager.hpp}
\DoxyCodeLine{    ├── Game}
\DoxyCodeLine{    │   └── IScene.hpp}
\DoxyCodeLine{    └── Graph}
\DoxyCodeLine{        ├── GraphStruct.hpp}
\DoxyCodeLine{        ├── IMusic.hpp}
\DoxyCodeLine{        ├── ISprite.hpp}
\DoxyCodeLine{        └── IText.hpp}
\DoxyCodeLine{}
\DoxyCodeLine{6 directories, 16 files}

\end{DoxyCode}
 uml\+: \href{https://github.com/ARCADE-TEK-2026/archi_arcade/blob/gh-pages/uml.svg}{\texttt{ https\+://github.\+com/\+ARCADE-\/\+TEK-\/2026/archi\+\_\+arcade/blob/gh-\/pages/uml.\+svg}} full documentation\+: \href{https://github.com/ARCADE-TEK-2026/archi_arcade/blob/main/documentation/uml.dox}{\texttt{ https\+://github.\+com/\+ARCADE-\/\+TEK-\/2026/archi\+\_\+arcade/blob/main/documentation/uml.\+dox}}

Comme vous pouvez le voir, nous avons decider d\textquotesingle{}integrer un ECS directement dans l\textquotesingle{}architecture commune.

Le projet se compose en 3 parties\+:
\begin{DoxyItemize}
\item Le Core (le binaire arcade)
\item Les jeux (sous forme de .so)
\item Les lib graphic (sous forme de .so)
\end{DoxyItemize}

\DoxyHorRuler{0}
 LE CORE

À implementer\+: 
\begin{DoxyCode}{0}
\DoxyCodeLine{\textcolor{comment}{/*}}
\DoxyCodeLine{\textcolor{comment}{  La classe principal pour le binaire arcade.}}
\DoxyCodeLine{\textcolor{comment}{  Elle doit avoir au moins le main menu, le IGameModule du jeux}}
\DoxyCodeLine{\textcolor{comment}{  actuel (si chargé), le IDisplayModule de la librairie actuel,}}
\DoxyCodeLine{\textcolor{comment}{  et un IEventManager qui sera passé en parametre}}
\DoxyCodeLine{\textcolor{comment}{  à la partie graphique et à la partie jeux.}}
\DoxyCodeLine{\textcolor{comment}{  À chaque frame, il faut appeler la méthode `update`.}}
\DoxyCodeLine{\textcolor{comment}{  La méthode `update` doit:}}
\DoxyCodeLine{\textcolor{comment}{  -\/ regarder si des évènements survenus dans le IEventManager:}}
\DoxyCodeLine{\textcolor{comment}{     -\/ demande à changer de lib graphique}}
\DoxyCodeLine{\textcolor{comment}{     -\/ demande à changer de lib de jeux}}
\DoxyCodeLine{\textcolor{comment}{     -\/ demande à quiter le jeux}}
\DoxyCodeLine{\textcolor{comment}{     -\/ demande à quiter le programme}}
\DoxyCodeLine{\textcolor{comment}{  -\/ si un jeux est chargé}}
\DoxyCodeLine{\textcolor{comment}{      -\/ appeler le `update` du IGameModule}}
\DoxyCodeLine{\textcolor{comment}{  -\/ sinon}}
\DoxyCodeLine{\textcolor{comment}{      -\/ appeler le `update` du main menu}}
\DoxyCodeLine{\textcolor{comment}{  -\/ appeler le `update` du IDisplayModule}}
\DoxyCodeLine{\textcolor{comment}{*/}}
\DoxyCodeLine{\textcolor{keyword}{class }\mbox{\hyperlink{a00888}{Arcade::Core::ICore}};}
\DoxyCodeLine{}
\DoxyCodeLine{\textcolor{comment}{/*}}
\DoxyCodeLine{\textcolor{comment}{  Le gestionneur des évènements survenu dans toute}}
\DoxyCodeLine{\textcolor{comment}{  l'éxecution du programme.}}
\DoxyCodeLine{\textcolor{comment}{*/}}
\DoxyCodeLine{\textcolor{keyword}{class }Arcade::ECS::IEventmanager;}

\end{DoxyCode}


\DoxyHorRuler{0}
 LES JEUX

Utilisation de sprite communs, soit présents dans l\textquotesingle{}archi commune dans assets/, soit à ajouter en pr.

À implementer dans chaque jeux\+: \`{}\`{}\`{}cpp /$\ast$ L\textquotesingle{}unité la plus petite de l\textquotesingle{}ECS, elle possède des composant. C\textquotesingle{}est la relation composant-\/system(expliqué plus tard) qui permet à un ECS de réaliser des actions. 