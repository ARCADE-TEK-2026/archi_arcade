Bienvenu sur cette Architecture Commune de l\textquotesingle{}\mbox{\hyperlink{a00060}{Arcade}}. Projet de 2ème année au sein de Epitech.

Source Tree\+: 
\begin{DoxyCode}{0}
\DoxyCodeLine{include/}
\DoxyCodeLine{├── Api.hpp}
\DoxyCodeLine{└── \mbox{\hyperlink{a00060}{Arcade}}}
\DoxyCodeLine{    ├── ArcadeStruct.hpp}
\DoxyCodeLine{    ├── Core}
\DoxyCodeLine{    │   ├── ICore.hpp}
\DoxyCodeLine{    │   ├── IDisplayModule.hpp}
\DoxyCodeLine{    │   └── IGameModule.hpp}
\DoxyCodeLine{    ├── ECS}
\DoxyCodeLine{    │   ├── IComponent.hpp}
\DoxyCodeLine{    │   ├── IEntity.hpp}
\DoxyCodeLine{    │   ├── IEntityManager.hpp}
\DoxyCodeLine{    │   ├── IEventManager.hpp}
\DoxyCodeLine{    │   ├── ISystem.hpp}
\DoxyCodeLine{    │   └── ISystemManager.hpp}
\DoxyCodeLine{    ├── Game}
\DoxyCodeLine{    │   └── IScene.hpp}
\DoxyCodeLine{    └── Graph}
\DoxyCodeLine{        ├── GraphStruct.hpp}
\DoxyCodeLine{        ├── IMusic.hpp}
\DoxyCodeLine{        ├── ISprite.hpp}
\DoxyCodeLine{        └── IText.hpp}
\DoxyCodeLine{}
\DoxyCodeLine{6 directories, 16 files}

\end{DoxyCode}


Comme vous pouvez le voir, nous avons decider d\textquotesingle{}integrer un ECS directement dans l\textquotesingle{}architecture commune.

Le projet se compose en 3 parties\+:
\begin{DoxyItemize}
\item Le Core (le binaire arcade)
\item Les jeux (sous forme de .so)
\item Les lib graphic (sous forme de .so)
\end{DoxyItemize}

\DoxyHorRuler{0}
 LE CORE

À implementer\+: 
\begin{DoxyCode}{0}
\DoxyCodeLine{\textcolor{keyword}{class }\mbox{\hyperlink{a00888}{Arcade::Core::ICore}};            \textcolor{comment}{// La classe principal pour le binaire arcade.}}
\DoxyCodeLine{                                      \textcolor{comment}{// Elle doit avoir au moins le main menu, le IGameModule du jeux}}
\DoxyCodeLine{                                      \textcolor{comment}{// actuel (si chargé), le IDisplayModule de la librairie actuel,}}
\DoxyCodeLine{                                      \textcolor{comment}{// et un IEventManager qui sera passé en parametre}}
\DoxyCodeLine{                                      \textcolor{comment}{// à la partie graphique et à la partie jeux.}}
\DoxyCodeLine{                                      \textcolor{comment}{// À chaque frame, il faut appeler la méthode `update`.}}
\DoxyCodeLine{                                      \textcolor{comment}{// La méthode `update` doit:}}
\DoxyCodeLine{                                      \textcolor{comment}{// -\/ regarder si des évènements survenus dans le IEventManager:}}
\DoxyCodeLine{                                      \textcolor{comment}{//    -\/ demande à changer de lib graphique}}
\DoxyCodeLine{                                      \textcolor{comment}{//    -\/ demande à changer de lib de jeux}}
\DoxyCodeLine{                                      \textcolor{comment}{//    -\/ demande à quiter le jeux}}
\DoxyCodeLine{                                      \textcolor{comment}{//    -\/ demande à quiter le programme}}
\DoxyCodeLine{                                      \textcolor{comment}{// -\/ si un jeux est chargé}}
\DoxyCodeLine{                                      \textcolor{comment}{//     -\/ appeler le `update` du IGameModule}}
\DoxyCodeLine{                                      \textcolor{comment}{// -\/ sinon}}
\DoxyCodeLine{                                      \textcolor{comment}{//     -\/ appeler le `update` du main menu}}
\DoxyCodeLine{                                      \textcolor{comment}{// -\/ appeler le `update` du IDisplayModule}}
\DoxyCodeLine{}
\DoxyCodeLine{\textcolor{keyword}{class }Arcade::ECS::IEventmanager;     \textcolor{comment}{// Le gestionneur des évènements survenu dans toute}}
\DoxyCodeLine{                                      \textcolor{comment}{// l'éxecution du programme.}}

\end{DoxyCode}


\DoxyHorRuler{0}
 LES JEUX

À implementer dans chaque jeux\+: 
\begin{DoxyCode}{0}
\DoxyCodeLine{  \textcolor{keyword}{class }\mbox{\hyperlink{a00896}{Arcade::ECS::IEntity}};           \textcolor{comment}{// L'unité la plus petite de l'ECS, elle possède des composant.}}
\DoxyCodeLine{C\textcolor{stringliteral}{'est la relation composant-\/system(expliqué plus tard)}}
\DoxyCodeLine{\textcolor{stringliteral}{qui permet à un ECS de réaliser des actions.}}
\DoxyCodeLine{\textcolor{stringliteral}{}}
\DoxyCodeLine{\textcolor{stringliteral}{  class Arcade::ECS::IComponent;        // Un composant représente uniquement des variables que l'}on}
\DoxyCodeLine{vient ajouter à une entité.}
\DoxyCodeLine{}
\DoxyCodeLine{  \textcolor{keyword}{class }\mbox{\hyperlink{a00900}{Arcade::ECS::IEntityManager}};    \textcolor{comment}{// Un gestionneur d'entité.}}
\DoxyCodeLine{Il permet de facilité la recherche de certaines entités.}
\DoxyCodeLine{}
\DoxyCodeLine{  \textcolor{keyword}{class }\mbox{\hyperlink{a00908}{Arcade::ECS::ISystem}};           \textcolor{comment}{// Ce sont des classes qui sont appelées à chaque frame du}}
\DoxyCodeLine{jeux avec leur méthode `\mbox{\hyperlink{a00908_a3ead01729785e8e87224626ea7bbda20}{run}}`.}
\DoxyCodeLine{}
\DoxyCodeLine{  \textcolor{keyword}{class }\mbox{\hyperlink{a00912}{Arcade::ECS::ISystemManager}};    \textcolor{comment}{// Un gestionneur de systeme.}}
\DoxyCodeLine{Il permet de faciliter l\textcolor{stringliteral}{'appel de toutes les méthode `run`}}
\DoxyCodeLine{\textcolor{stringliteral}{de tous les systemes qu'}il posséde.}
\DoxyCodeLine{}
\DoxyCodeLine{  \textcolor{keyword}{class }\mbox{\hyperlink{a00920}{Arcade::Game::IScene}};           \textcolor{comment}{// Représente une étape de cinématique/jeux du jeux.}}
\DoxyCodeLine{La première fois qu\textcolor{stringliteral}{'elle est lancé, la méthode `init`}}
\DoxyCodeLine{\textcolor{stringliteral}{est appelée.}}
\DoxyCodeLine{\textcolor{stringliteral}{Quand la scene est quitté (on change de scene/...)}}
\DoxyCodeLine{\textcolor{stringliteral}{la méthode `close` est appelée.}}
\DoxyCodeLine{\textcolor{stringliteral}{}}
\DoxyCodeLine{\textcolor{stringliteral}{  class Arcade::Game::IGameModule;      // L'}\textcolor{keyword}{interface }par laquelle le binaire arcade inter-\/agit.}
\DoxyCodeLine{À chaque frame, la méthode `update` est appelée.}
\DoxyCodeLine{À chaque frame, la méthode `getCurrentEntityManager`}
\DoxyCodeLine{est appelée. Cela permet de passer les entitées qu'il}
\DoxyCodeLine{faut au IDisplayModule (qui lui va s'occuper d'afficher ce qu'il comprend)}
\DoxyCodeLine{}
\DoxyCodeLine{  class \mbox{\hyperlink{a00060}{Arcade}}::Graph::IMusic;          \textcolor{comment}{// Des composants "{}normalisés"{} qui permettent}}
\DoxyCodeLine{  class \mbox{\hyperlink{a00948}{Arcade::Graph::IText}};           \textcolor{comment}{// au IDisplayModule d'affiché/de jouer}}
\DoxyCodeLine{  class \mbox{\hyperlink{a00944}{Arcade::Graph::ISprite}};         \textcolor{comment}{// ces composants.}}

\end{DoxyCode}


\DoxyHorRuler{0}
 LES LIB GRAPHIC

À implementer dans chaque lib graphic\+: 
\begin{DoxyCode}{0}
\DoxyCodeLine{\textcolor{keyword}{class }\mbox{\hyperlink{a00936}{Arcade::Graph::IDisplayModule}};  \textcolor{comment}{// L'interface par laquelle le binaire arcade inter-\/agit.}}
\DoxyCodeLine{                                      \textcolor{comment}{// Il permet de faire un rendu graphique des composant}}
\DoxyCodeLine{                                      \textcolor{comment}{// "{}normalisés"{} (IText, IMusic, ISprite)}}
\DoxyCodeLine{                                      \textcolor{comment}{// À chaque frame, la méthode `update` est appelée.}}
\DoxyCodeLine{                                      \textcolor{comment}{// Il est attendu à ce que cette méthode mette à jour}}
\DoxyCodeLine{                                      \textcolor{comment}{// les évènements survenus.}}
\DoxyCodeLine{}
\DoxyCodeLine{                                      \textcolor{comment}{// Pour vous faciliter la vie, vous pouvez avoir la même architecure que dans le IGameModule:}}
\DoxyCodeLine{                                      \textcolor{comment}{// Un ISystemManager qui manage vos différents systemes.}}
\DoxyCodeLine{                                      \textcolor{comment}{// Dans ce cas:}}
\DoxyCodeLine{}
\DoxyCodeLine{\textcolor{keyword}{class }\mbox{\hyperlink{a00908}{Arcade::ECS::ISystem}};           \textcolor{comment}{// regardez les infos plus haut.}}
\DoxyCodeLine{\textcolor{keyword}{class }Arcade::ECS::ISystemMAnager;    \textcolor{comment}{// regardez plus haut.}}

\end{DoxyCode}


\DoxyHorRuler{0}


N.\+B.\+: L\textquotesingle{}écrivain de cette documentation prie de bien vouloir l\textquotesingle{}excuser pour la prise de partie, il sera heureux de débatre par l\textquotesingle{}intermédiaire d\textquotesingle{}issue sur github dans le cas où une information dans cette documentation serait fausse. N.\+B.\+: Pour get des éléments dans un vecteur const il faut utiliser la méthode .at()

lien du github\+: \href{https://github.com/ARCADE-TEK-2026/archi_arcade}{\texttt{ https\+://github.\+com/\+ARCADE-\/\+TEK-\/2026/archi\+\_\+arcade}}

\DoxyHorRuler{0}


 